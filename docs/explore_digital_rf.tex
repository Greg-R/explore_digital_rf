

\documentclass[oneside,letterpaper,12pt]{book}
\usepackage[utf8]{inputenc}
\usepackage[T1]{fontenc}
\usepackage[top=0.8874in,bottom=0.8874in,left=0.7874in,right=0.7874in]{geometry}
\usepackage{charter}
\usepackage{color}
\usepackage{array,ragged2e}
\usepackage{graphicx}
\usepackage[colorlinks=true, linkcolor=red]{hyperref}
\usepackage{amsmath}
\usepackage{titling,lipsum}
\usepackage{booktabs}
\usepackage{longtable}
\usepackage{float}
\usepackage{parskip}

%opening
\title{}
\author{}

\begin{document}
	
\include{cover/cover}

\frontmatter
Fun with GNU Radio and Digital RF

Copyright 2018 by Gregory Raven
%\maketitle
\tableofcontents
\listoftables
\listoffigures
	

\mainmatter

\chapter{Setting up Python Environments}

Ubuntu 18.04 has Python 3.7.  To install:

sudo apt-get install python3.7

The binary is installed here:

/usr/bin/python3.7

\section{Create a Python Environment}

python3.7 -m venv py37

To launch the environment:

cd py37/bin
source activate

This should appear to the left of the usual bash prompt:

(py37)

To exit the environment:

deactivate

Will need to install virtualenv, as Python 2.7 does not have venv.

sudo apt-get install virtualenv

\section{Install Spyder IDE}

As of October 2018, Spyder does not build with Python 3.7.
Create a Python 3.6 env and enter it.

\begin{verbatim}
(py36) bash_prompt$  pip3.6 install spyder
\end{verbatim}

\section{Prerequisites for GNU Radio}

Here is the most detailed ``build guide'' for GNU Radio:

\url{https://wiki.gnuradio.org/index.php/BuildGuide}

The guide has a link to another page with various distributions and their prerequisites.
Here is the link for Ubuntu:

\url{https://wiki.gnuradio.org/index.php/UbuntuInstall}

Note that the link to Debian doesn't contain any relevant install information.
However, since Ubuntu is a derivative of Debian, the Ubuntu prerequisite list
should be all or mostly good.

Install the following using apt-get:

\begin{verbatim}
sudo apt-get -y install git cmake g++ python-dev swig  \
pkg-config libfftw3-dev libboost-all-dev libcppunit-dev libgsl-dev \
libusb-dev libsdl1.2-dev python-wxgtk3.0 python-numpy python-cheetah \
python-lxml doxygen libxi-dev python-sip libqt4-opengl-dev libqwt-dev \
libfontconfig1-dev libxrender-dev python-sip python-sip-dev python-qt4 \
python-sphinx libusb-1.0-0-dev libcomedi-dev libzmq3-dev python-mako \
python-gtk2
\end{verbatim}


\section{Special Requirements for Pygtk and PyQt4}



\section{Something that forces Python 2.7, what was it???}

\section{Installation of GNU Radio}

GNU Radio has a large number of dependencies.  Due to the difficulty of installing everything correctly, a special script is available which does most of the work.

It is called ``PyBOMBS'':

\url{https://www.gnuradio.org/blog/pybombs-the-what-the-how-and-the-why}

Forget about trying to install inside a Python environment.  It is possible to
accomplish, but there are problems with some of the dependencies which make
this hard to accomplish.

\subsection{Using the Distribution GNU Radio Package}

It is easy enough on Debian and Ubuntu to install the bundled-up version of GNU Radio:

sudo apt-get install gnuradio

Typically these distribution software packages can be well tested, but older versions.
In my tests, I found that the current Ubuntu repository installs version 3.7.11.  Debian is installing 3.7.10.
A manual install described below results in version 3.7.13.4.

I typically like to use the latest-and-greatest.  However, be warned that manual install
 and build of GNU Radio may not proceed easily.  It's a very complex software suite with
 numerous dependencies.

\subsection{Debian/Ubuntu GNU Radio Development Package}

sudo apt-get install gnuradio-dev

\subsection{Pybombs Installation Script}

GNU Radio has so many dependencies, that a Python installer script was developed.
It is also possible to install with the usual "Make" tools.  It appears that even
this is difficult, and Pybombs was developed specifically for the requirements of
GNU Radio and installation or skipping of possible features or variants.

The Pybombs script allows for creating of multiple ''prefixes'', so installation
of variants is easy to accomplish.  This serves the same purpose as having multiple
environments.

Using the Python install tool ``pip'':

pip install pybombs

Next, make a directory.  This will become the ``prefix'' directory:

mkdir gnuradio1

Now use pybombs to initialize the prefix:

pybombs prefix init gnuradio1

The prefix is now loaded with GNU Radio default recipes:

pybombs -p gnuradio1 recipes add-defaults

Build and install GNU Radio to the prefix:

pybombs -p gnuradio3 install gnuradio

The build and installation process will take a long time, perhaps an hour or more.

\section{Installation of the SDRPlay Driver}

\section{Installation of gr-sdrplay}

\section{Installation of Digital\_RF}

Installation of the Digital\_RF Package is relatively easy.

The easiest way to install is via pip:

\begin{verbatim}
pip install digital_rf
\end{verbatim}

This may install the blocks files in an unusual location:

\begin{verbatim}
/home/(username)/.local/share/gnuradio/grc/blocks
\end{verbatim}

This path needs to be added to the file config.conf as follows.
If this file was not created during installation, create it file here:

\begin{verbatim}
/home/(userid)/.gnuradio/config.conf
\end{verbatim}

\begin{verbatim}
[grc]
local_blocks_path = /home/(username)/.local/share/gnuradio/grc/blocks
\end{verbatim}

\backmatter

\end{document}









